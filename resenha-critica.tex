\documentclass[a4paper,oneside]{article}


\pagestyle{myheadings}

%%%%%%%%%%%%%%%%%%%%%%%%%%%
% Pacotes para acentua��o %
%%%%%%%%%%%%%%%%%%%%%%%%%%%
%\usepackage{cite}
%\usepackage{natbib}     %% note the absence of options
% Change how references are included (see the natbib package).
%\setcitestyle{square,numbers,authoryear}
%\setcitestyle{authoryear}
\usepackage[alf]{abntex2cite}
\usepackage[utf8]{inputenc}
\usepackage[T1]{fontenc}
\usepackage{ae}
\usepackage{booktabs}
%\usepackage[ansinew]{inputenc}
\usepackage{graphicx}

\usepackage[brazilian]{babel}

%%%%%%%%%%%%%%%%%%%%%%%%%%%%%%%%%%%%%%%%%%%%%%%%%%%
\usepackage{algpseudocode,algorithm}
% Declaracoes em Português
\algrenewcommand\algorithmicend{\textbf{fim}}
\algrenewcommand\algorithmicdo{\textbf{faça}}
\algrenewcommand\algorithmicwhile{\textbf{enquanto}}
\algrenewcommand\algorithmicfor{\textbf{para}}
\algrenewcommand\algorithmicif{\textbf{se}}
\algrenewcommand\algorithmicthen{\textbf{então}}
\algrenewcommand\algorithmicelse{\textbf{senão}}
\algrenewcommand\algorithmicreturn{\textbf{devolve}}
\algrenewcommand\algorithmicfunction{\textbf{função}}

% Rearranja os finais de cada estrutura
\algrenewtext{EndWhile}{\algorithmicend\ \algorithmicwhile}
\algrenewtext{EndFor}{\algorithmicend\ \algorithmicfor}
\algrenewtext{EndIf}{\algorithmicend\ \algorithmicif}
\algrenewtext{EndFunction}{\algorithmicend\ \algorithmicfunction}

% O comando For, a seguir, retorna 'para #1 -- #2 até #3 faça'
\algnewcommand\algorithmicto{\textbf{até}}
\algrenewtext{For}[3]%
{\algorithmicfor\ #1 $\gets$ #2 \algorithmicto\ #3 \algorithmicdo}

%%%%%%%%%%%%%%%%%%%%%%%%%%%%%%%%%%%%%%%%%%%%%%%%%%%


\usepackage{nomencl}
\makenomenclature
\renewcommand{\nomname}{Lista de Símbolos}


\linespread{1.5} % espa�amento entre linhas



% horizontal
\setlength{\hoffset}{-1in}

\setlength{\oddsidemargin}{3.0cm} 

\setlength{\textwidth}{160mm} % (210mm - 30mm - 20mm)

\setlength{\parindent}{1.25cm} % identa��o de cada par�grafo

% vertical
\setlength{\voffset}{-1in}
\addtolength{\voffset}{2.0cm}

\setlength{\topmargin}{0.0cm}

\setlength{\headheight}{5mm}
\setlength{\headsep}{5mm}

\setlength{\textheight}{247mm} % (297mm - 30mm - 20mm)



\title{Respostas slides sobre programação linear}

\author{Lucas Campos Dal Piaz de Souza \\
Mestrado de Engenharia de Produção e Sistemas Computacionais\\

Universidade Federal Fluminense

} 



\begin{document}


\pagenumbering{arabic}

\maketitle

%\citeonline{nome_referenciado no refs}  \cite{nome_referenciado no refs}


%\thispagestyle{empty} 

\section*{Exemplo 4}
\begin{center}
    Minimizar  $3x_1 + 6x_2 + 5x_3 + 6x_4$ \\
    Sujeitos à \\
  $8x_1 + 3x_2 + 5x_3 + 6x_4 \leq 15000$ \\
  $5x_1 + 7x_2 + 4x_3 + 5x_4 \leq 20000$ \\
  $12x_1 + 9x_2 + 8x_3 + 7x_4 \leq 40000$ \\
  $x_1, x_2, x_3, x_4 \geq 0$\\
\end{center}

\section*{Exemplo 5}
\begin{center}
    Maximizar  $0,23x_1 + 0,3x_2 + 0,42x_3 $ \\
    Sujeitos à \\
  $3,5x_1 + 5x_2 + 6,7x_3 \leq 150$ \\
  $x_1 + \frac{4}{3}x_2 + \frac{5}{3}x_3 \leq 40$ \\
  $x_1 + x_2 + x_3 \leq 30$ \\
  $x_1, x_2, x_3 \geq 0$\\
\end{center}

\section*{Exemplo 6}

\begin{center}
  Minimizar  $0,5x_1 + 0,18x_2 + 0,2x_3 + 0,16x_4 + 0,3x_5 + 0,1x_6$ \\
  Sujeitos à \\
  $225x_1 + 364x_2 + 337x_3 + 385x_4 + 15x_5 + 42x_6 \leq 3200$ \\
  $7x_1 + 0x_2 + 2x_3 + 0x_4 + 87x_5 + 13x_6 \leq 750$ \\
  $0x_1 + 0x_2 + 3x_3 + 0x_4 + 12x_5 + 59x_6 \leq 70$ \\
  $2,9x_1 + 1,3x_2 + 7,9x_3 + 0,1x_4 + 1,3x_5 + 0,7x_6 \leq 10$ \\
  $11x_1 + 9x_2 + 86x_3 + 0x_4 + 43x_5 + 34x_6 \leq 650$ \\
  $x_1, x_2, x_3, x_4, x_5, x_6 \geq 0$\\
  
    
\end{center}

\section*{Exemplo 7}

\begin{center}
$x_{1j}$ corresponde ao abastecimento da região São paulo I para os cinco destinos, com j de 1 até 5\\
$x_{2j}$ corresponde ao abastecimento da região São paulo II para os cinco destinos, com j de 1 até 5\\
$x_{3j}$ corresponde ao abastecimento da região Perímetros irrigados do NE para os cinco destinos, com j de 1 até 5\\
  Minimizar  $52x_{11} + 77x_{12} + 241x_{13} + 466x_{14} + 444x_{15}$ \\
   $60x_{21} + 85x_{22} + 249x_{23} + 473x_{24} + 452x_{25}$ \\
    $110x_{31} + 135x_{32} + 191x_{33} + 501x_{34} + 477x_{35}$ \\
  Sujeitos à \\
  $52x_{11} + 77x_{12} + 241x_{13} + 466x_{14} + 444x_{15} \leq 771$ \\
  $60x_{21} + 85x_{22} + 249x_{23} + 473x_{24} + 452x_{25} \leq 964$ \\
  $110x_{31} + 135x_{32} + 191x_{33} + 501x_{34} + 477x_{35} \leq 193$ \\
  $18(x_{11} + x_{21} + x_{31}) + 7(x_{12} + x_{22} + x_{32}) + 1680(x_{13} + x_{23} + x_{33}) + 159(x_{14} + x_{24} + x_{34}) + 64(x_{15} + x_{25} + x_{35}) \leq 1928$ \\
  $9(x_{11} + x_{21} + x_{31}) + 4(x_{12} + x_{22} + x_{32}) + 840(x_{13} + x_{23} + x_{33}) + 70(x_{14} + x_{24} + x_{34})  + 32(x_{15} + x_{25} + x_{35})\leq 964$ \\
  $x_{ij} \geq 0$\\
  
    
\end{center}





\end{document}



